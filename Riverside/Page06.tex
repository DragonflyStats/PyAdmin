Why should a sample be random

Avoids, bias ( a systematic tendency to overestimate or underestimate the population parameters)

With a random sample, statistical techniques can be used to make probability statements about the population parameter



Bias Consistent repeated divergence of the sample from the population parameter in the same direction, caused by a tendency for 
a certain group of the population to be omitted from the sampling scheme of its subject who refuse to co-operate form a group which is, in
some way, different to the populations

%---------------------------------%



Error in Sampling

\begin{itemize}
\item Bias
\item Lack of Precision i.e. in repeated sampling, the values of the sample statistic are spread out or scattered; the result of sampling is not 
repeatable
\end{itemize}

%--------------------------------%

Increases the sample size increases the precision of sample statistics
