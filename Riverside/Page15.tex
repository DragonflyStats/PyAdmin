Test Scores Examples
Number of corrent Answers (out of 30) for 25 Student.
%====================================================%
\begin{center}
\begin{tabular}{|ccccc|
\hline
25 & 29 & 23 & 27 & 25 \\
23 & 22 & 25 & 22 & 28 \\
28 & 28 & 17 & 24 & 30 \\
19 & 17 & 23 & 21 & 24 \\
15 & 20 & 26 & 19 & 23 \\
\hline
\end{tabular}
\end{center}
%------------------------------------%
\begin{itemize}
\item Divide the range of the variable into a number of mutuall exclusive intervals (class intervals) and count the number falling into each interval.
\item 
Choose Intervals such that no interval can fall on the boundary, i.e. use class intervals whose end-points are not possible values for the variable in question.
\item 
Each measurement should fall into one and only one subinterval.
\end{itemize}


%--------------------------------------%




How many class intervals?
Usually we would have at least 5, but as many as 20. One rule of thumb is to select a number near the square root of the sample size.

For our example, we will use 6 intervals

\[  \mbox{ClassWidth} =  \frac{\mbox{Largest Value} - \mbox{Smallest Value} }{\mbox{No. Of Classes}} \]


\[ \mbox{ClassWidth} =  \frac{30 - 15}{6} \approx 3\]

Class size andwidth are essentially arbitary. The only rule in groupng data is that the classes must be mutually exclusive (each piece of data is placed into one and only one class) and all-inclusive (all data must be included).



%=================================================%


Class Interval & Frequencu & Rel. Freq & Percentage \\


%===============================================%
