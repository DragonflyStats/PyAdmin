%====================================================%
25 & 29 & 23 & 27 & 25 \\
23 & 22 & 25 & 22 & 28 \\
28 & 28 & 17 & 24 & 30 \\
19 & 17 & 23 & 21 & 24 \\
15 & 20 & 26 & 19 & 23 \\


Divide the range of the variable into a number of mutuall exclusive intervals (class intervals) and count the number falling into each interval.

Choose Intervals such that no interval can fall on the boundary


Each measurement should fall into one and only one subinterval.


How many class intervals?
Usually we would have at least 5, but as many as 20. One rule of thumb is to select a number near the square root of the sample size.

For our example, we will use 6 intervals


\[  \frac{30 - 15}{6} \approx 3\]



Class Interval & Frequencu & Rel. Freq & Percentage \\


%===============================================%
