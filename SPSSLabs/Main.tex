MA4505     Computer Lab 2:   Introduction to SPSS
 
 
SPSS is a statistical software package. To access SPSS, click on Start, then Programs, then SPSS Inc, then SPSS 16.0, then SPSS 16.0 again. In SPSS, the data editor window will open and you will be asked if you want to run the tutorial, open an existing file etc. We want to enter a new data set so select Type in data and click OK.
 
 
\subsection{The Data Editor :}
The Data Editor provides a convenient, spreadsheet-like environment for creating and editing SPSS data files. The Data Editor window opens automatically when you start an SPSS session. You can create new data files or modify existing ones with the Data Editor. In the data editor:
 
1.      Rows represent the data from one unit (person/thing). For example, each individual respondent to a questionnaire is a unit.
2.      Columns represent variables. For example, each question on a questionnaire is a variable.
3.      Each cell contains a single value of a variable for a unit. Cells only contain data values. Unlike spreadsheet programs, cells in the Data Editor cannot contain formulas.
 

\subsection{Views in SPSS} 
There are two views for the data editor – the data view and the variable view. When setting up a new data set, first we need to define our variables and this is done in the variable view. Then we need to input our data and this is done in the data view.
 
We want to create a data set which contains the data collected from 25 students in UL using the following questionnaire:
 
 
 
 
 
 
 
 
 
%===========================================================================================% 
\subsection*{Questionnaire}
 
This questionnaire is anonymous. Please answer the following questions by ticking the appropriate box or filling in your answer in the space provided.
 
 
Q1.  Are you       male  „
                        female              „
 
 
image Q2.  How long is it (approximately in weeks) since your last haircut?
 
 
image Q3. How much (approximately in euros) did you spend on your last haircut?
 
 
Q4. Do you work part-time during the semester?   Yes   „
                                                                        No    „
 
 
Q5. Rate your satisfaction with your choice of degree on a scale of 0 to 10 where 0 is not at all satisfied and 10 is extremely satisfied.
image
 
Q6. In a typical week, how often would you exercise?
 
Never     „
Once or twice a week  „
          Several times a week  „
          Every day    „
 
Q7. In a typical week, how often would you drink alcohol?
 
Never     „
Once or twice a week  „
          Several times a week  „
          Every day    „
 
 
Q8. What is the best thing, for you, about being in college?
 
image
 
image
 
Thank you.
 
 
 
%=============================================================================% 
 
 
 
 
Each question in our questionnaire will generate a variable so we have 8 variables. For the moment, we will ignore the data generated by the open question at the end of the questionnaire and concentrate on the following variables generated by question 1 to question 7:
 
\item[Gender] (coded as 1=male, 2=female)
\item[HcutTime] (length of time since last haircut)
\item[HcutCost] (amount spent on last haircut)
\item[Work] (whether a student works part-time coded as 1=yes and 2=no)
\item[Satisfaction]
\item[Exercise] (coded as 0=never, 1=once or twice a week, 2= several times a week, 3 = every day)
\item[Alcohol] (coded as 0=never, 1=once or twice a week, 2= several times a week, 3 = every day)
 
 
To define each variable, click on variable view at the bottom left of your screen. In the variable view window, you need to give each variable a name, a type (numeric, string (text), date etc.), a width (how many spaces you need), number of decimal places, label (to describe the variable if the variable name is not self explanatory) and values (assign codes to text).
 
In row 1 type in gender under Name. The type is Numeric, the width is 8, the decimal places are 0. The label can be left blank since gender is self explanatory. Gender is a qualitative variable so we need to define our codes for male and female. For values, click on None and click on the grey box to the right of None. In the box that opens, enter 1 for value and male for value label. Click add. Enter 2 for value and female for value label. Click add. Click OK.
 
In row 2 type in HcutTime under Name. This is a quantitative variable so we don’t need to define codes. Under Label type in time since last haircut in weeks to give more information on the variable. None of the other columns need to be changed.
 
Define the rest of the variables for the dataset using the information on them at the top of the page. Give appropriate label types and values where needed.
 
When the variables are defined, click on data view at the bottom left of the screen. Each question in our questionnaire has generated a variable (or column) and each person’s questionnaire will generate a row of data. Data from the first three questionnaires is as follows:
 
Gender
HcutTime
HcutCost
Work
Satisfaction
Exercise
Alcohol
1
8
12
2
6
2
0
1
2
15
2
7
0
1
1
1
10
1
5
1
1
Once you have inputted this data, click on View in the menu at the top of the screen and make sure that value labels is ticked. Your data should now look like:
 
Gender
HcutTime
HcutCost
Work
Satisfaction
Exercise
Alcohol
male
8
12
no
6
several times a week
never
male
2
15
no
7
never
once or twice a week
male
1
10
yes
5
once or twice a week
once or twice a week
 
Each row corresponds to the data from a questionnaire completed by an individual (unit).
%===============================================================================================% 
 
\subsection{Summarising the data}
 
The rest of the data has been inputted in the file which was emailed to you. Open this file using File from the main menu, then Open and then Data. 
Find the location and name of the file you saved onto your memory stick.
 
%===============================================================================================% 
\subsection{Describing a qualitative variable}
 
Gender is a qualitative variable. To describe it, select Analyze from the main menu and then Descriptive Statistics and then Frequencies. 
Click on gender and click on the arrow to select it. Click on charts and click on pie charts. Click Continue and click OK.
 
An output window appears with a frequency distribution for gender and a pie chart.
 
 
 
image
 
image
 
 
 
 
Use the same menu option to answer the following questions:
 
What percentage of the 25 students in the class work part-time?
 
 
 
What percentage of the 25 students in the class never drink alcohol?
 
 
 
What percentage of the 25 students in the class exercise every day?
 
Sometimes we want to describe two qualitative variables together so that we can answer questions like are males in the class more likely to work part-time than females? 
Select Analyze from the main menu and then Descriptive Statistics and then Crosstabs. 
Click on the name of one of the qualitative variables e.g. gender and click on the arrow to select it in the rows. 
Click on the name of the other qualitative variable e.g. work and click on the arrow to select it in the columns. Click on Cells and select row and column percentages. Click Continue and click OK.
 
image
 
% within gender gives percentages of the row data and % within work gives percentages of the column data. Of the 18 males, 11 (61.1%) work part-time. 
Of the 7 females, 6 (85.7%) work part-time.
 
Use the same menu option to answer the following questions:
 
Are males in the class more likely to exercise every day than females?
 
 
 
Can you see any relationship between frequency of exercising and frequency of alcohol consumption e.g. are those who never exercise more likely to drink frequently?
 
%=======================================================================================% 
\subsection{Describing a quantitative variable}
 
HcutTime is a quantitative variable. To describe it, select Analyze from the main menu and then Descriptive Statistics and then Explore. 
Select HcutTime in the Dependent List. Click on Plots and tick Histogram. Click continue. Click OK. 
The output includes a boxplot, all the summary statistics e.g. mean, median etc. and a histogram.
image
 
 
 
 
How would you describe the shape of the histogram? What measure of centrality and variability would you use?
 
 
 
Experiment with these menu options and describe the other variables.
 
 
We can also describe a quantitative variable by groups e.g. describe the satisfaction rating of males and females. 
Select Analyze from the main menu and then Descriptive Statistics and then Explore. Select the name of the quantitative variable in the Dependent List e.g. satisfaction. 
Select the name of the grouping or qualitative variable in the Factor List e.g. gender. Click on Plots and tick Histogram. Click continue. Click OK.
 
Males, on average, are slightly more satisfied with the degree than females with a mean of 7.176 compared to a mean of 7 for females.
To save your output (graphs etc.), select File in the output window, then Save as and then type in the name. All output files have a .spv extension in SPSS.
 
To save your data file, select File in the data editor window, then Save as and then type in the name. All data files have .sav extension in SPSS.
 
%===============================================================================================================================%
\newpage 
 
 
 
 
 
 
 
 
 
 
 
 
MA4505   Computer Lab 3 Sampling and testing for normality
 
\section{Sampling}

\begin{itemize} 
\item This section uses some of the theory from Section 1.4 of your notes (revise the definitions on page 6).
\item To access SPSS, click on Start, then Programs, then SPSS Inc, then SPSS 16.0, then SPSS 16.0 again. In SPSS, the data editor window will open and you will be asked if you want to run the tutorial, open an existing file etc. Click cancel.
\item In computer lab 1, we used Excel to select a random sample from the population. Here we will use SPSS to randomly select units of the population. For example, we might want to select a sample of size 10 from a population of size 50. Each unit of the population is given a number (from 0 to 49). Open up a new data file in SPSS by selecting File from the main menu, then New and then Data. Input the numbers 0 to 49 in one column called population. Select Data from the Main menu, then Select Cases. Select Random sample of cases. Click on Sample and click on Exactly. We want to select exactly 10 cases from the first 50 cases. Click Continue. In the Output box, select Delete unselected cases. Click OK. A sample of 10 cases has been randomly selected – units in the population with these numbers make up your random sample.
\item Lets go back to the data from computer lab 2 i.e. the data collected from 25 students. Select File from the main menu, then Open, then Data. Enter the name of the file you saved in Computer Lab 2 and where the file is located. Click Open.
\itemDoes random sampling work i.e. does the statistic from a random sample provide a good estimate of the population parameter? Lets treat the 25 students who filled out the questionnaire as a population (all students at a lecture when the questionnaire was handed out). We are interested in the following parameter: percentage of this population who work part-time. We can find out the value of this parameter by using the menu options from Computer Lab 2 to describe the variable work. Select Analyze from the main menu and then Descriptive Statistics and then Frequencies. Click on the variable work and click on the arrow to select it. Click OK.
\end{itemize}
 
image
 
68% of the population work part-time. Here, the value of the population parameter is known but most of the time, we have to estimate it using the value of a statistic calculated from a sample of data. We will take a sample of size 15 from our population.
Select Data from the Main menu, then Select Cases. Select Random sample of cases. Click on Sample and click on Exactly. We want to select exactly 15 cases from the 25 cases. Click Continue. In the Output box, there is the option to filter out the cases (units) that are not in your sample, copy the selected cases to a new dataset or delete the unselected cases. Filtered cases remain in the data file but are excluded from analysis. Filtered cases are indicated with a slash through the row number in the Data Editor window. We will select Filter out unselected cases. Click OK.

On the left-hand side of the data editor window, you should see a line through some of the row numbers. The data from these units are not included in the analysis. Only 15 rows of data are included.

Now calculate the percentage of this sample of 15 who work part-time. Select Analyze from the main menu and then Descriptive Statistics and then Frequencies. Click on the variable work and click on the arrow to select it. Click OK.
Everyone’s random sample will be different but you should see a similar table to this:

image
 
The sample size is 15 and 66.7% of the sample work part-time. This is a very good estimate of the population parameter we calculated earlier (68%).

\section{Testing for normality}
 
This section uses some of the theory on page 19-23 and page 42-44 of your notes. In computer lab 2 we described quantitative data using graphs e.g. histogram, boxplots and we also calculated numerical summaries e.g. mean, median, standard deviation etc. How do we decide which numerical summaries to use for a particular set of data? First we must decide if the data is normally distributed. If the data is normally distributed we will use the mean as the measure of centrality and the standard deviation as the measure of variability. If the data is not normally distributed we will use the median as a measure of centrality and the interquartile range as a measure of variability. How do we check if the data is normally distributed? We can compare the values of the mean and median (they should be similar for a normal distribution), we can look at the shape of the histogram (which should be reasonably bell-shaped and symmetric for a normal distribution), we can also use more formal tests of normality and normal probability plots.
 
The amount of money spent on advertising for 18 companies is as follows:
 
      5000 2500      500     10000     2000    250   20000     6000     250
      500 750      6000    7000 4000   2000   1000      3000    2000   
Enter the data into one column in SPSS called amount. Select Analyze from the main menu and then Descriptive Statistics and then Explore. Select amount in the Dependent List. Click on Plots and tick Histogram. Tick Normality Plots with Tests. Click continue. Click OK.
 
Is the variable amount normally distributed? The histogram is positively skewed. There is an outlier in the boxplot (observation 7 with a value of 20000). The mean (4041.66) is greater than the median (2250). image
 
 
Not all the points are close to the line in the normal probability plot which indicates that the data may not be normally distributed.
 
Tests of Normality
 
 
 
 	
Kolmogorov-Smirnov(a)
Shapiro-Wilk
Statistic
df
Sig.
Statistic
df
Sig.
amount
.216
18
.025
.739
18
.000
a  Lilliefors Significance Correction
 
 
The Shapiro-Wilk test is preferred when the sample size (n) is less than 50. The p-value from the test (called \texttt{Sig.}) is .000 (written as <0.001). The p-value gives the probability that the data come from a normal distribution – if the probability is very small (<0.05) then we reject the idea that the data are normally distributed and conclude that the data are not normally distributed. Here the probability is so small (< 0.001), we conclude that the data is not normally distributed.
 
Sometimes when we get non-normal data, we can change the scale of our data i.e. transform it to get a normal distribution. One transformation that often works for positively skewed data is the natural log(ln) transformation.
 
Select Transform from the main SPSS menu, then Compute Variable. This option is a calculator with many different functions in SPSS. The target variable is the new column of data that will be created containing the natural log of amount e.g. lnamount. The numeric expression is the formula used to calculate it e.g. ln(amount).
 
Is the transformed variable normally distributed? Use the histogram, boxplot, values of mean, median, normal probability plot and test for normality to answer this question.
 
image
 
 
The probability that our data come from a normal distribution is 0.635 > 0.05 so we can conclude that our data are normally distributed. We can use the mean of the transformed data as a measure of centrality and the standard deviation as a measure of variability. Transformations don’t always make non-normal data normally distributed but are worth trying because many statistical techniques make the assumption that the data are normally distributed.
 
 
Use the dataset from computer lab 2 to do the following exercises.
\begin{enumerate}
\item Using the normal probability plot and normality tests, is the data for hcuttime normally distributed?
\item Using the normal probability plot and normality tests, is the data for hcutcost normally distributed?
\item Using the normal probability plot and normality tests, is the data for satisfaction for females normally distributed?
\end{enumerate} 
 
 
 
Solutions: Both hcuttime and hcutcost are not normally distributed (p<0.001) from the normality tests, mean and median very different, positively skewed histograms and not all points close to the line in the normal probability plots.
 
 
The histogram of satisfaction for females is given below:
 
image
 
 
%-----------------------------------------------------% 
Tests of Normality
 	
gender
Kolmogorov-Smirnova
Shapiro-Wilk
 	
Statistic
df
Sig.
Statistic
df
Sig.
satisfaction
Male
.259
17
.004
.869
17
.022
Female
.214
7
.200*
.858
7
.144
a. Lilliefors Significance Correction
 	 	 	 
*. This is a lower bound of the true significance.
 	 	 
 
The probability that the data for satisfaction for females is normally distributed =0.144 > 0.05 so we can conclude that the data is normally distributed and we can use the mean as a measure of centrality and the standard deviation as a measure of variability. The data for satisfaction for males is not normally distributed (p=0.022 < 0.05) so we would use the median as the measure of centrality and the interquartile range as a measure of variability.

%================================================================================================================% 
MA4505   Computer lab 4
 
The first section of this computer lab asks you to answer questions on a new dataset called \textbf{spending.sav} using SPSS and the menu options in Computer Lab 2 and 3. The data file contains information on 30 small companies. Data was collected on the following variables:
 
Sector: sector of the company where 1=manufacturing industry, 2=service industry.
Amount: amount spent by the company on advertising
Where: where the company advertises most where 1=TV, 2=radio, 3=newspaper
 
 
To answer each question, first you need to decide if the question involves one variable or two and also what type of data is generated by the variable(s). For example, Q1 involves one quantitative variable (amount), Q4 requires you to describe a quantitative variable (amount) by a qualitative variable (sector).
 
 
Try the following questions and if you want feedback on your answers, hand up the lab sheet to the lab assistant.
 
 
ID number:      Name:
 
 
1. Is the data for amount spent on advertising normally distributed? Justify your answer.
 
 
 
 
 
 
 
2. Give a suitable measure of centrality and variability for the amount spent on advertising.
 
 
 
 
 
3. What percentage of companies advertises mostly in newspapers?
 
 
 
 
4. Compare the amount spent on advertising by companies in manufacturing and companies in the service industry i.e. give a measure of centrality for each and a measure of variability. Which sector has the most variability in amount spent on advertising?
 
 
 
 
 
 
 
 
5. What percentage of companies in the service industry advertised mainly on the radio?
 
 
 
 
%============================================================================================% 
 
\subsection{Confidence Intervals}

%%- This section covers some of the concepts on page 50-55 of your notes.

1. Let us assume that the 30 companies in the spending.sav dataset are a population of companies. The mean of a variable such as amount spent on advertising is therefore, denoted by m. We can find out the value of m (m = €3023.06) using the following menu commands: select Analyze from the main menu and then Descriptive Statistics and then Explore. Select the name of the quantitative variable in the Dependent List i.e. amount. Click on Plots and tick Histogram. Tick Normality plots with tests. Click continue. Click OK.
The value of m is usually unknown so usually we have to estimate it using the mean of a sample randomly selected from the population. The mean of this sample (called image ) is the best estimate of m and 95% of the time, m lies in the range image where SE(image ) = image ( s is the sample standard deviation and n is the sample size).

Select Data from the Main menu, then Select Cases. Select Random sample of cases. Click on Sample and click on Exactly. We want to select exactly 15 cases from the first 30 cases. Click Continue. Click OK. A random sample of 15 companies has been selected from the population of all 30 companies.

Find the mean of the variable amount for this sample. The mean of this sample is the best estimate of m. A 95% confidence interval for m is also given in the output. 95% of the time, m lies in this range. Everyone’s random sample will be different and will give different results – here are the results from one randomly selected sample of size 15.

Descriptives
 	 	 	
Statistic
Std. Error
amount
Mean
3058.6487
66.68721
95% Confidence Interval for Mean
Lower Bound
2915.6188
 
Upper Bound
3201.6785
 
5% Trimmed Mean
3066.9207
 
Median
3090.4300
 
Variance
66707.756
 
Std. Deviation
258.27845
 
Minimum
2562.99
 
Maximum
3405.41
 
Range
842.42
 
Interquartile Range
435.80
 
Skewness
-.482
.580
Kurtosis
-.533
1.121
The mean of this sample is €3058.65 which is a good estimate of the true mean €3023.06. A 95% confidence interval for m is given by [2915.62, 3201.68] which includes the true value of m=€3023.06.

What is the estimate of m from your sample and does your confidence interval include m i.e. €3023.06?

Your estimate of m:

Your 95% confidence interval:

Take another random sample from the population, this time of size 20. Use SPSS to get the mean of this sample and a 95% confidence interval. Does your confidence interval include m?

Your estimate of m:

Your 95% confidence interval:

Compare your confidence interval to the first one. What has happened to the confidence interval? As n increases, the width of the confidence interval decreases i.e. we get more precise.

Calculate the mean of this sample again but this time calculate a 99% confidence interval. Select Analyze from the main menu and then Descriptive Statistics and then Explore. Select the variable amount. Select Statistics and change the confidence level for the mean to 99%. Click continue and click OK.

Your estimate of m:

Your 99% confidence interval:

Compare this confidence interval to the 95% confidence interval for the same sample. What has happened to the confidence interval? As the level of confidence increases, the width of the confidence interval increases i.e. we get less precise.

 

2. A machine which packs sugar gives a normal distribution of weights of filled packets with a mean of 1000g as long as the machine is working properly. A sample of 20 packets are selected and weighed. The sample data are as follows:
 
\begin{framed}
\begin{verbatim} 
1000 990 995 998 1023 1015 999 997 1009  995 1010
1003 1002 1000 999 998 995 1003 989 1010
\end{verbatim}
\end{framed}
 
\begin{itemize}
\item 
Open a new data file in SPSS and enter the data in a column called weight. Select Analyze from the main menu and then Descriptive Statistics and then Explore. Summarise the variable weight.
\item  
Give your best estimate of m, where m - mean weight of all filled packets:
\item  
Use the confidence interval to decide if the machine is working correctly.
\item  
Is the machine working correctly? Justify your answer.
\end{itemize}
 
%============================================================================================% 
 
 
MA4505  Computer Lab 5  Hypothesis Testing
 
 
This computer lab carries out a one sample hypothesis test for the mean and a hypothesis test for paired data. The relevant section of your book of notes is Section 5 page 59 to page 63.
 
Example 1:
The width of a certain population is normally distributed with mean 23 i.e. m=23. The widths of a sample of 10 observations were found to be 28, 21, 26, 16, 18, 13, 15, 22, 19, 22. Does this sample come from a population with mean width =23?
 
Input the data into a column in SPSS. Call the column width. Select Analyze from the main menu and then Descriptive statistics and then Explore. Select the variable width and click OK.
 
 
Descriptives
 	 	 	
Statistic
Std. Error
width
Mean
20.0000
1.50555
95% Confidence Interval for Mean
Lower Bound
16.5942
 
Upper Bound
23.4058
 
5% Trimmed Mean
19.9444
 
Median
20.0000
 
Variance
22.667
 
Std. Deviation
4.76095
 
Minimum
13.00
 
Maximum
28.00
 
Range
15.00
 
Interquartile Range
7.25
 
Skewness
.263
.687
Kurtosis
-.633
1.334
 
 
 
 
 
The mean of the sample (image =20) is the best estimate of the mean of the population (m). We are 95% confident that m lies in the range [16.6, 23.4]. This interval includes the test value of 23, therefore it is possible that this sample comes from a population with m=23.
 
Let’s carry out a hypothesis test to investigate if this sample comes from a population with m=23.
 
Null hypothesis: m=23
Alternative hypothesis: m ¹ 23
 
Select Analyze from the main menu and then Compare Means and then One-sample T test. Select the variable width.  The test value is 23. Click OK. The output is as follows:
 
image
 
image
 
%-------------------------------------------------% 
The difference between the sample mean and the test value is –3 i.e. the value of our sample mean=20 minus the value of the population mean=23 and we are 95% confident this difference lies in the interval [-6.4, 0.4]. This confidence interval includes zero i.e. no difference between the sample mean and the test value of 23 so it is possible that this sample comes from a population with m = 23. The value of the test statistic t is -1.993. If you were doing this hypothesis test by hand, you would compare this value with critical values from the t tables. The critical value from the t tables would be ± 2.262 (not given in computer output).
 
 
Since –1.99 is not less than –2.26 Þ do not reject the null hypothesis. Instead of using critical values from tables, SPSS gives you the P-value i.e. the probability of getting these results when the null hypothesis is true. If the p-value is < 0.05 then reject the null hypothesis, otherwise do not reject. Our p-value (called Sig. 2-tailed in the output) is 0.077 > 0.05 Þ do not reject the null hypothesis. We conclude that this sample comes from a population where the mean width is 23.
 
Note: we will always come to the same conclusion from our confidence interval and hypothesis test.
 
 
 
Example 2:
 
For the data set from computer lab 2 quest.sav, test the following hypotheses using the same menu options as above.
\begin{itemize} 
\item[(a)]     the students come from a population with a mean of 7 for satisfaction rating with their choice of degree
\item[(b)]     the students come from a population with a mean of €10 for cost of last haircut.
\end{itemize}
 
For (a) show that the sample mean is 7.125 and 95% of the time, m lies in the range [6.62, 7.63]. This range includes 7 so it is likely these students come from a population where the mean is 7.
 
Null hypothesis: m=7
Alternative hypothesis: m ¹ 7
 
Show that the p-value from the hypothesis test is 0.612 > 0.05 so we do not reject the null hypothesis that these students come from a population with a mean of 7.
 
For (b) show that the sample mean is 18.92 and 95% of the time, m lies in the range [8.84, 29.00]. This range includes 10 so it is likely that these students come from a population with a mean of 10.
 
Null hypothesis: m=10
Alternative hypothesis: m ¹ 10
 
Show that the p-value from the hypothesis test is 0.08 > 0.05 so we do not reject the null hypothesis that these students come from a population with a mean of 10.
 
%============================================================================================% 
  
Paired Data
Suppose a shoe company wanted to compare two materials, A and B, for use on the soles of childrens’ shoes. A paired design allows you to remove individual variation i.e. within a child so you can see more clearly any differences between the materials you are studying. In the paired design, each child would wear a special pair of shoes with the sole on one shoe made from Material A (MatA) and the sole on the other shoe from Material B (MatB). After three months, the shoes worn by 10 children were measured for wear and the data are as follows:
 
 
MatA:  13.2  8.2  10.9  14.3  10.7  6.6  9.5  10.8  8.8  13.3
MatB:  14.0  8.8  11.2  14.2  11.8  6.4  9.8  11.3  9.3  13.6
 
Input the data into two columns in SPSS called MatA and MatB.
 
We are interested in the difference between the two columns. Select Transform from the main menu and then Compute Variable. The target variable is the new variable we want to create so call the target variable diff. The numeric expression is how we create the target variable which is the difference between MatA and MatB i.e. MatA – MatB. Click OK and a new variable has been created in your dataset called diff. We want to investigate these differences so summarise this new variable diff. Select Analyze from the main menu and then Descriptive statistics and then Explore. Select Plots and tick histogram and normality tests with plots.
 
 
Tests of Normality
 	
Kolmogorov-Smirnova
Shapiro-Wilk
 	
Statistic
df
Sig.
Statistic
df
Sig.
diff
.188
10
.200*
.961
10
.801
a. Lilliefors Significance Correction
 	 	 
*. This is a lower bound of the true significance.
 	 
 
 
The p-value from the normality test is 0.801 > 0.05 so we conclude that the differences are normally distributed.
 
 
 
 
 
 
 
 
 
 
 
 
 
 
 
 
 
 
Descriptives
 	 	 	
Statistic
Std. Error
diff
Mean
-.4100
.12243
95% Confidence Interval for Mean
Lower Bound
-.6870
 
Upper Bound
-.1330
 
5% Trimmed Mean
-.4056
 
Median
-.4000
 
Variance
.150
 
Std. Deviation
.38715
 
Minimum
-1.10
 
Maximum
.20
 
Range
1.30
 
Interquartile Range
.45
 
Skewness
-.078
.687
Kurtosis
.108
1.334
 
 
The mean difference (MatA – MatB) is –0.41 so the wear for material A tends to be less than the wear for Material B. We are 95% confident that the true difference lies in the range -0.687 to -0.133. Zero (i.e. no difference) does not lie in the confidence interval so we can conclude there is a statistically significant difference between the two materials used for the shoes. Also the confidence interval for MatA – MatB contains all negative values so the wear for material A tends to be less than the wear for Material B.
 
We would like to carry out a hypothesis test to test the claim that there is no difference between the two materials. The null hypothesis is that the mean difference in wear between these two materials equals zero (md = 0 ) and the alternative hypothesis is the mean difference is not equal to zero (md ¹ 0 ).
 
The assumption made for a hypothesis test on paired data is that the differences are normally distributed which we have already shown to be true.
 
To carry out a hypothesis test on the paired data, select Analyze from the main menu, then Compare Means and then Paired-Samples T test. Select MatA for Variable 1 and MatB for Variable 2. Click OK.
image
 
The p-value from the test is 0.009 <0.05 so we reject the null hypothesis and conclude that there is a significant difference between the wear for the two materials.
 
 
 
 
 
 
 
 
 
Summary: Describe the variable of interest for your sample, get a measure of centrality e.g. mean and measure of variability e.g. standard deviation. The mean of the sample (image ) is the best estimate of the mean of the population (m)
where m is unknown. Give a range of values you are 95% confident includes the value of m i.e. a 95% confidence interval using the output. Test claims about m i.e. is m = some test value or not. Use the p-value to decide if you reject the claim that m = some test value. If the p-value < 0.05, we reject the claim or hypothesis that m = some test value.
 
If you have two paired measurements, get the difference between them. Get the mean difference, get a 95% confidence interval for the difference, test the claim that the mean difference is zero i.e. there is no difference between the two sets of measurements. If the p-value < 0.05, we reject the claim or hypothesis that md = 0.
 
%============================================================================================% 
  
 
 
MA4505 Science Maths 3 Computer Lab 6
 
\subsection{Comparing the means from two independent groups}
 
The relevant section of your notes is pages 64-67. Open the data file quest.sav. We are interested in comparing the mean of the variable satisfaction for two groups (males and females). These groups are independent of one other. Let m1 = mean of satisfaction for males and let m2 = mean of satisfaction for females.
 
H0 : m1 = m2 or m1 - m2 = 0
Ha : m1 ¹ m2 or m1 - m2 ¹ 0
 
 
Select Analyze from the main menu, then Compare Means and then Independent Samples T test. The test variable is the quantitative or numeric variable i.e. satisfaction and the grouping variable is the qualitative variable that puts the data into two groups i.e. gender. Select satisfaction as the test variable and gender as the grouping variable. Click on Define Groups. Males were coded as 1 and females were coded as 2 so the groups are defined by 1 for group 1 and 2 for group 2. Click continue. Click OK.
 
image
 
The summary statistics above show that there were 17 males and 7 females. The mean of satisfaction was higher for the males than the females.  Standard deviations from two different groups cannot be compared directly because they are calculated with different means.The coefficient of variation defined by (standard deviation/mean)* 100 is used to compare the variation of the two samples. The coefficient of variation (not given in the output) is higher for males (18.6%) compared to females (11.7%) indicating more variability in the satisfaction of males than females.
 
image
 
 
 
There are two rows of output – one assuming equal variances and one not assuming equal variances. Because we have small samples (both less than 30), we need to check the assumption of equal variances.
H0 : s12= s22
Ha : s12¹  s22
 
The p-value (called Sig. in the output) from the Equality of Variances test is 0.307 > 0.05 so we do not reject the null hypothesis and can assume equal variances. We will, therefore, use the output from the row of the table assuming equal variance. The best estimate of the difference between the means for the two groups i.e. mean for satisfaction for males minus mean for satisfaction for females is 0.17647 indicating that males have higher satisfaction ratings than females on average. We are 95% confident the difference lies in the range [-0.955, 1.308]. This range includes zero (no difference) so there isn’t a statistically significant difference between the means of the two groups.

\begin{description} 
H$_0$ : m1 = m2
H$_1$ : m1 ¹ m2
\end{description} 
The p-value from the t test for equality of means is 0.749 > 0.05 so we do not reject the null hypothesis and conclude there isn’t a statistically significant difference between the means of the two groups.
 
 
 
2. Using the same data file and menu options, compare the means of HcutTime for those who work part-time and those who don’t (work part time group is coded as 1 and don’t work part-time group is coded as 0). The mean is higher for those that work part-time (9.35 weeks compared to 8.37 weeks). Show that we can assume equal variances (p-value =0.82 > 0.05). The best estimate of the difference between the means of the two groups is –0.98. We are 95% confident the difference lies in the range –10.97 to 9.02. This interval contains zero so there is no significant difference between the means of the two groups. The p-value for the t test for the equality of means is 0.841 > 0.05 so we do not reject H0 : m1 = m2 and conclude there is no statistically significant difference between the means of the two groups.
 
 
 
Investigating the relationship between two quantitative (numeric) variables
 
The relevant section of your notes is from page 73-77. Open the data file pointsqca.sav. The data consists of points in the Leaving Cert for 28 students in UL and their QCA at the end of first year.
 
 
 
The first step in describing the relationship between points and QCA is to draw a scatter plot. Select Graphs from the main menu, then Chart builder.
Click OK. Select a Scatter/dot diagram and double click on the very first plot called a simple scatter plot. Close the Elements Properties Window. With your mouse, drag the points variable to the X-axis and the QCA variable to the Y-axis. Click OK to get the following scatter plot:
image
 
There seems to be a weak positive relationship between points and QCA. As points go up, QCA tends to go up but there are observations that don’t fit that pattern. The plot shows one obvious outlier with a value of 570 for points and a value of 0.57 for QCA (observation 18).
 
To measure the strength of the relationship between points and QCA, we need to calculate the value of the correlation coefficient. Select Analyze, Correlate and Bivariate. Select points and QCA. Click OK.
 
image
 
 
 
The correlation coefficient (r) has a value of 0.326 which is positive so as points go up, QCA tends to go up but it is not a very strong relationship ($-1 \leq r \leq 1$). The null hypothesis is that r = 0 i.e. there is no relationship. We do not reject that hypothesis since the p-value = 0.09 > 0. 05.
 
 
Delete the obvious outlier (observation 18) by clicking on row 18 (grey cell numbered 18) and pressing delete. Investigate the effect of this outlier being removed by redoing the scatter plot and calculating the correlation coefficient.
 
image
There is a stronger relationship between points and QCA when the outlier is removed (r = 0.562). We reject the null hypothesis that there is no relationship (r=0) since the p-value =0.002 < 0.05 and conclude that there is a statistically significant relationship between points and QCA.
 
Outliers cannot be deleted unless it can be established that they are as a result of a data entry error or some other obvious reason. Usually, the analysis with and without the outlier is presented to illustrate the effects of the removal of the outlier.
 
 
 
\end{document} 
